\documentclass[12pt]{article}

\usepackage{xcolor}
\usepackage{listings}
\usepackage{amsmath}
\usepackage{graphicx}
\usepackage{setspace}
\usepackage{hyperref}


\begin{document}

\title{OS9}
\date{December 2020}
\maketitle

\section{text}


  \large \itshape why \LaTeX ? \\
\small I It makes beautiful documents \\
 I Especially mathematics  \\
 I It was created by scientists, for scientists  \\
 I A large and active community  \\
 I It is powerful — you can extend it  \\
 I Packages for papers, presentations, spreadsheets, . . .
\end{center}

\section{image}
\begin{center}
 \includegraphics[scale=1]{1.jpg}  \\
  \caption{best nature picture}   
\end{center}
  

\section{table}
\begin{tabular}{|r|1|}
  \hline
  7C0 & hexadecimal \\
  3700 & octal \\ \cline{2-2}
  11111000000 & binary \\
  \hline \hline
  1984 & decimal \\
  \hline
\end{tabular}

\section{formula}
\begin{equation}\label{eqn:einstein}
E=mc^2\tag{*}
\end{equation}
% \eqref{eqn:einstein}

\begin{align*}
\sin A \cos B &= \frac{1}{2}\left[ \sin(A-B)+\sin(A+B) \right] \\
\sin A \sin B &= \frac{1}{2}\left[ \sin(A-B)-\cos(A+B) \right] \\
\cos A \cos B &= \frac{1}{2}\left[ \cos(A-B)+\cos(A+B) \right] \\
\end{align*}


\begin{theorem}
Use the Stirling formula to show that
\begin{equation*}
\frac{a^n}{n!}\sim\frac{1}{\sqrt{2\pi n}}
\left(\frac{ae}{n}\right)^n~~~(n\to\infty)
\end{equation*}
when a is any constant, and deduce that
\[ \lim_{n \to \infty}\frac{a^n}{n!}=0\]
\end{theorem}

\section{code}
\begin{latin}
\begin{lstlisting}[language=C]
#include <stdio.h>
#define N 10
/* Block
 * comment */

int main()
{
    int i;

    // Line comment.
    puts("Hello world!");

    for (i = 0; i < N; i++)
    {
        puts("LaTeX is also great for programmers!");
    }

    return 0;
}
\end{lstlisting}
\end{latin}

\end{document}
